\section{Pseudorandom Function and Pseudorandom Permutation}

이 장에서는 Pseudorandom Generator(PRG)와 Pseudorandom Function(PRF) 그리고
Pseudorandom Permutation(PRP)을 정의합니다.

\subsection*{Pseudorandom Generators}

$\prg$를 길이 $n$의 문자열을 길이 $l(n) > n$의 출력에 매핑하는 효율적으로 계산
가능한 결정론적 함수라고 합시다. 효율적인 구분자 $\adv$가 $\prg$에 의해 출력된
문자열이 주어지는지 아니면 임의로 균일하게 선택된 문자열이 주어지는지 구분할 수
없는 경우 $\prg$를 PRG라고 부릅니다.

\begin{figure}[ht]
  \centering
  \begin{tikzpicture}
    [inner sep=0.5cm, every text node part/.style={align=left}]
    % experiment players and what players are doing
    \node at (0, 0) (chl) {challenger $\chl$};
    \node at (7, 0) (adv) {adversary $\adv$};

    % draw lines
    \draw[<->] (chl) -- (adv)
    node[above, midway, inner sep=0.1cm] {\textsf{PRG} experiment};

    % challenger C
    \node at (0, -1) (b) {$b \unisamp \binset{}$};
    \node at (0, -3) (samp_r) {
      if $b = 1$ \\
      \quad $s \unisamp \binset{n}$ \\
      \quad $r := \prg(\seed)$ \\
      otherwise \\
      \quad $r \unisamp \binset{\len(\klen)}$
      };
    \node at (0, -5) (get_bp) {$b'$};
    \node at (0, -6) (return) {return $b \issame b'$};

    % adversary A
    \node at (7, -3) (get_r) {$r$};
    \node at (7, -5) (chs_bp) {choose $b' \in \binset{}$};

    % draw lines
    \draw[->] (samp_r) -- (get_r);
    \draw[<-] (get_bp) -- (chs_bp);
  \end{tikzpicture}
  \caption{Indistinguishability experiment for a PRG}
  \label{fig:exp_prg}
\end{figure}

공식적인 정의는 다음과 같습니다.

\begin{definition} 
  Let $\prg$ be a DPT algorithm such that for any $\klen$ and any input $\seed \in
  \binset{\klen}$, the result $\prg(\seed)$ is a string of length $\len(\klen)$. $\prg$ is a
  \emph{pseudorandom generator} if the following conditions hold:
  \begin{itemize}
      \item (Expansion.) For every $\klen$ it holds that $\len(\klen) > \klen$. We call $\len(\klen)$ the \emph{expansion factor} of $\prg$.
      \item (Pseudorandomness.) For any PPT algorithm $\adv$, there is a negligible function $\negl$ such that
            \begin{equation}
                \pr{\expprg_{\adv, \prg}(n) = 1} \leq \frac{1}{2} + \negl(n).
            \end{equation}
  \end{itemize}
  \label{def:prg}
\end{definition}

\subsection*{Pseudorandom Functions}
균일한 키 $\key \in \binset{\klen}$에 대한 효율적이고 결정론적인 키 함수 
$\prf: \binset{\inlen} \to \binset{\outlen}$가, 동일한 정의역과 치역을 가지는 모든 
함수의 집합 $\rfspace$에서 균일하게 임의로 선택된 함수 $\rf$와 구별할 수 없는 경우, $\prf$를 
PRF라고 합니다. 공식적인 정의를 위해 다음과 같이 실험 \ref{fig:exp_prf}를 구성합니다. 이 실험에서 
오라클 $\orc$은 결정론적입니다.

\begin{figure}[ht]
  \centering
  \begin{tikzpicture}
    [inner sep=0.5cm, every text node part/.style={align=left}]
    % experiment players and what players are doing
    \node at (0, 0) (chl) {challenger $\chl$};
    \node at (7, 0) (adv) {adversary $\adv$};

    % draw lines
    \draw[<->] (chl) -- (adv)
    node[above, midway, inner sep=0.1cm] {\textsf{PRF} experiment};

    % challenger C
    \node at (0, -1) (key_gen) {$k \samp \gen(1^n)$};
    \node at (0, -2) (b) {$b \unisamp \binset{}$};
    \node at (0, -4) (get_x) {
      if $b = 1$ \\
      \quad $y \samp \prf(x)$ \\
      otherwise \\
      \quad $y \samp \rf(x)$
      };
    \node at (0, -6) (get_bp) {$b'$};
    \node at (0, -7) (return) {return $b \issame b'$};

    % adversary A
    \node at (7, -1) (get_n) {$1^n$};
    \node at (7, -4) (chs_x) {choose $x$};
    \node at (7, -6) (chs_bp) {choose $b' \in \binset{}$};

    % draw lines
    \draw [->] (key_gen) -- (get_n);
    \draw[red, <->] (chs_x) -- (get_x)
    node [midway, above, inner sep=0.1cm] {query $x$}
    node [midway, below, inner sep=0.1cm] {response $y$};
    \draw[<-] (get_bp) -- (chs_bp);

  \end{tikzpicture}
  \caption{Indistinguishability experiment for a PRF}
  \label{fig:exp_prf}
\end{figure}

$\rfspace$에서 $\rf$를 균등하게 뽑는 과정은 효율적이지 않습니다. 따라서 적절한 실험 구성을 위해 
알고리듬 $\ref{alg:rf}$와 같이 on-the-fly 방식으로 $\rf$를 구현합니다.

\begin{algorithm}
  \begin{algorithmic}[1]
    \Require $x$
    \Ensure $y$
    \Procedure{$\rf(x)$}{}
    \If{$\exists (x, y') \in \tb$}
    \State $y \samp y'$
    \ElsIf{$\nexists (x, y') \in \tb$}
    \State $y \unisamp \binset{n}$
    \State $\tb \samp \tb \cup (x, y)$
    \EndIf
    \State \Return $y$
    \EndProcedure
  \end{algorithmic}
  \caption{Random Function (on-the-fly)}
  \label{alg:rf}
\end{algorithm}

실험 \ref{fig:exp_prf}을 바탕으로 PRF를 정의합니다.

\begin{definition} \label{def:prf}
  An efficient, keyed function
  $\prf: \binset{\inlen} \to \binset{\outlen}$ is a \emph{pseudorandom function} 
  if for all PPT distinguishers $\adv$, there is a negligible function $\negl$ 
  such that:
  \begin{equation}
    \pr{\expprf{\adv, \prf}(n) = 1} \le \frac{1}{2} + \negl(n).
    \label{equ:prf}
  \end{equation}
\end{definition}

명제 \ref{pro:same_def}와 비슷한 방법으로, 식 \ref{equ:prf}을 다음과 같이 바꿀 수 있습니다.
\begin{equation}
  \abs{\pr{\adv^{\prf}(1^n) = 1} - \pr{\adv^{\rf}(1^n) = 1}} \le \negl(n).
  \label{equ:prf2}
\end{equation}

\subsection*{Pseudorandom Permutations}
$\prp$가 PRF이면서 전단사 함수라면, 우리는 $\prp$를 PRP라고 합니다. 즉, 균일한 키 
$\key \in \binset{\klen}$에 대한 효율적이고 결정론적인 키 함수 
$\prp: \binset{\len} \to \binset{\len}$가, 동일한 정의역과 치역을 가지는 모든 일대일
함수(순열)의 집합 $\rpspace$에서 균일하게 임의로 선택된 함수 $\rp$와 구별할 수 없는 경우, 
$\prp$를 PRP라고 합니다. PRP의 정의 및 실험은 PRF와 유사하므로 생략하며, $\rp$를 on-the-fly 
방식으로 구현한 알고리듬만 소개합니다.
\begin{algorithm}
  \begin{algorithmic}[1]
    \Require $x$
    \Ensure $y$
    \Procedure{$\rp(x)$}{}
    \If{$\exists (x, y') \in \tb$}
    \State $y \samp y'$
    \ElsIf{$\nexists (x, y') \in \tb$}
    \While{$\nexists (x', y) \in \tb$}
    \State $y \unisamp \binset{n}$
    \EndWhile
    \State $\tb \samp \tb \cup (x, y)$
    \EndIf
    \State \Return $y$
    \EndProcedure
  \end{algorithmic}
  \caption{Random Permutation (on-the-fly)}
  \label{alg:rp}
\end{algorithm}

$\prp$가 키 순열이라면 $\prp$를 기반으로 한 암호 체계는 $\prp$ 자체를 계산하는 것 외에도 
$\iprp$를 계산해야 할 수도 있습니다. 이는 잠재적으로 새로운 보안 문제를 야기할 수 있습니다. 
특히, 이제는 구분자에게 역 순열에 대한 오라클 접근 권한이 추가로 주어져도 $\prp$와 $\rp$을 구별할 수 
없다는 더 강력한 요구 사항을 부과해야 할 수도 있습니다. $\prf$가 이러한 속성을 가지고 있다면, 우리는 
이를 strong PRP(SPRP)이라고 부릅니다. 즉, 다음의 식을 만족하는 negligible 함수 $\negl$가 
존재한다면, $\prp$는 SPRP입니다.
\begin{equation}
  \abs{\pr{\adv^{\prp, \iprp}(1^n) = 1} - \pr{\adv^{\rp, \irp}(1^n) = 1}}
  \le \negl(n).
\end{equation}
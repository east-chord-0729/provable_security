\section{Modes of Operation and Encryption}
% 암호 체계 \ref{def:prg_otp} 및 \ref{def:prf_otp}에 설명된 암호화 체계에는 실제 응용 
% 프로그램에 적합하지 않은 여러 가지 단점이 있습니다. 우선 구성 암호 체계 \ref{def:prg_otp}는 
% IND-PASS-security만 제공합니다. 또한 두 구성 모두 고정 길이 메시지의 암호화를 위해서만 정의됩니다. 
% 암호 체계 \ref{def:prf_otp}는 섹션 3.4.3의 끝에서 설명한 접근 방식을 사용하여 임의 길이 메시지를 
% 암호화하는 데 사용할 수 있지만 암호문 길이가 평문 길이의 일정한 배수인 체계는 다소 비효율적입니다. 이 
% 섹션에서는 이러한 단점을 극복하는 방법을 보여줍니다.

% 실용적인 고려 사항을 다루는 동안, 우리는 안전한 암호화 체계의 기본 구성 요소, 즉 의사난수 생성기와 
% 의사난수 순열이 각각 스트림 암호와 블록 암호를 사용하여 실제 세계에서 인스턴스화되는 방법에 대해서도 
% 논의하기 시작합니다. 여기서 우리의 목표는 주로 적절한 용어와 구문을 소개하는 것입니다. 스트림 암호와 
% 블록 암호가 어떻게 설계되는지, 그리고 그러한 기본 요소에 대한 몇 가지 인기 있는 후보에 대한 심층적인 
% 논의를 7장(Practical Constructions of Symmetric-Key Primitives)으로 미룬다.

\subsection*{Stream Ciphers and Stream-Cipher Modes of Operation}

% 정의 3.14와 같은 의사난수 생성기 G는 출력 길이가 고정되어 있기 때문에 다소 융통성이 없습니다. 따라서 
% G는 구조 3.17을 적용하여 임의 길이의 메시지를 처리하기에 적합하지 않습니다. 특히 G가 확장 인자 ℓ을 
% 가지고 있다고 가정해 보겠습니다. 우리는 G를 사용하여 길이가 ℓ' > ℓ인 메시지를 단일 n비트 키로 쉽게 
% 암호화할 수 없습니다. 그리고 G의 출력을 잘라냄으로써 길이가 ℓ' < ℓ인 메시지를 암호화할 수 있지만, 
% 이렇게 하는 것은 ℓ 의사난수 비트를 생성한 다음 그 중 ℓ - ℓ'를 폐기하는 것을 수반하기 때문에 낭비가 
% 됩니다.

% 스트림 암호는 의사난수 생성기를 인스턴스화하는 데 실제로 사용되며, 더 큰 유연성을 제공합니다. 스트림 
% 암호의 출력 비트는 점진적으로 그리고 필요에 따라 생성되므로 응용 프로그램이 필요한 만큼의 의사난수 비트를 
% 정확히 요청할 수 있습니다. 이렇게 하면 (생성할 수 있는 비트 수에 상한이 없기 때문에) 유용성이 확장되고 
% (외부 의사난수 비트가 생성되지 않기 때문에) 효율성이 향상됩니다.

Formally, a stream cipher is a pair of deterministic algorithms
$(\stinit, \stnext)$ where:
\begin{itemize}
  \item $\stinit$ takes as input a seed sand an optional initialization vector
        $\iv$, and outputs some initial state $\st$.
  \item $\stnext$ takes as input a current state $\st$ and outputs a bit $y$
        along with updated state $\st'$.
        % 실제로 넥스트는 한 번에 하나의 비트만 출력하는 것이 아니라 바이트나 심지어 더 많은 수의 
        % 무작위 비트를 출력할 수도 있습니다. 여기서는 단순화를 위해 비트를 출력한다고 가정합니다.
\end{itemize}

We define an algorithm $\getbits$ that takes as input an initial state $\st_0$
and a desired output length $1^l$ and then does:
\begin{itemize}
  \item For $i = 1$ to $l$, compute $(y_i,\st_i) := \stnext(\st_{i - 1})$.
  \item Return the $l$-bit string $y = y_1 \cdots y_l$ as well as the final
        state $\st_l$.
\end{itemize}
We let $\getbits_1$ be the algorithm that runs $\getbits$ and only returns
its initial output (namely, the $l$-bit string $y$).

% IV가 없는 보안 스트림 암호는 단지 유연한 인터페이스를 가진 의사난수 생성기일 뿐입니다. 즉, 균일한 
% 시드에서 Init을 실행하여 st0을 얻은 다음 GetBits1을 사용하여 (다항식) 비트 수를 생성하면 결과 
% 출력은 의사난수가 됩니다.

Given a stream cipher and a parameter $l = l(n) > n$, we may define the
deterministic function $\prg^l$ as
\begin{equation}
  \prg^l(\seed) := \getbits_1(\stinit(\seed), 1^l).
\end{equation}
Then the stream cipher is secure if $\prg^l$ is a pseudorandom generator for
any polynomial $l$.

% IV를 사용하는 스트림 암호의 보안은 여러 가지 방법으로 정의할 수 있습니다. 이 경우 보안은 의사난수 
% 함수의 보안과 유사하다고 정의합니다. 특히 여기서는 균일한 시드를 선택한 다음 Init(s,·)를 IV에 대해 
% 서로 다른 값을 사용하여 반복 실행하는 설정을 고려합니다. 요구 사항은 서로 다른 초기 상태를 사용하여 
% GetBits1을 실행하면 독립적으로 균일하게 보이는 출력 스트림이 생성되어야 한다는 것입니다.

Security for a stream cipher that does take an $\iv$ can be defined in multiple
ways. Given a stream cipher where $\stinit$ takes an $n$-bit $\iv$ and a
parameter $l = l(n)$, we may define the keyed function $F^l$ as
\begin{equation}
  F(\seed, \iv) = \getbits_1(\stinit(\seed, \iv), 1^l).
\end{equation}
Then the stream cipher is secure if $F^l$ is a pseudorandom function for any
polynomial $l$.

% 실제 스트림 암호는 일반적으로 임의의 n 값(시드의 길이와 IV를 결정함)을 지원하지 않고 일부 고정된 n 
% 값에 대해서만 작동합니다. 따라서 구체적인 보안 정의는 위에 주어진 점근적 정의보다 더 적합합니다.

% 의사난수 함수 F는 IV를 갖는 스트림 암호 (Init, Next)를 구성하는 데 사용될 수 있습니다. 기본 
% 아이디어는 스트림 암호의 시드 s를 F의 키로 사용하고, IV에 의해 결정된 값에서 시작하는 일련의 연속적인 
% 입력에 대한 Fs를 평가하는 것입니다.

% 스트림 암호는 이러한 방식으로 의사난수 함수로부터 구성될 수 있지만, 실제로 사용되는 스트림 암호의 전용 
% 구성은 일반적으로 특히 리소스가 제한된 환경에서 더 나은 성능을 갖습니다.

% 스트림 암호(Init, Next)를 사용하여 임의 길이의 메시지를 암호화하는 두 가지 동작 모드, 즉 동기화 
% 모드와 비동기화 모드에 대해 설명합니다.

% 스트림 암호는 종종 두 당사자 사이의 온라인 통신 세션을 암호화하는 데 사용됩니다. 이 경우 새로운 키 k가 
% 당사자에 의해 생성되고, 그 키는 세션 중에 전송된 메시지를 암호화하는 데 사용됩니다. 당사자 간의 통신이 
% 모든 메시지가 순서대로 도착하고 어떤 메시지도 손실되지 않도록, 두 당사자는 동기화되고 다음 방법을 
% 사용하여 송신자 S에서 수신자 R로 일련의 메시지를 암호화할 수 있습니다:

\newline

We discuss two modes of operation for encrypting arbitrary-length messages
using a stream cipher: synchronized mode and un-synchronized mode.

(synchronized mode) The two parties are synchronized and the following method
can be used to  encrypt a series of messages from a sender $\sender$ to a
receiver $\receiver$:
\begin{itemize}
  \item Both parties call $\stinit(k)$ to obtain the same initial state
        $\st_0$.
  \item Let $\st_{\sender}$ be the current state of $\sender$. If $\sender$
        wants to encrypt a message $\pt$, it computes
        $(y, \st'_{\sender}) := \getbits(\st_{\sender}, 1^{|\pt|})$, sends
        $\ct := \pt \xor y$ to the receiver, and updates its local state to
        $\st'_{\sender}$.
  \item Let $\st_{\receiver}$ be the current state of $\receiver$. When
        $\receiver$ receives a ciphertext $\ct$ from the sender, it computes
        $(y, \st'_{\receiver}) := \getbits(\st_{\receiver}, 1^{|\ct|})$,
        outputs the message $\pt := \ct \xor y$, and updates its own local
        state to $st'_{\receiver}$.
\end{itemize}
% 위의 설명에서 동일한 당사자는 항상 발신자 역할을 합니다. 그러나 당사자는 두 번째 키를 공유함으로써 
% 양방향 통신을 지원할 수 있습니다. 세션 중에 암호화된 모든 메시지의 총 합계 길이를 ℓ로 표시합니다. 
% 개념적으로 동기화 모드 암호화는 (1) ℓ을 미리 고정할 필요가 없고 (2) 전체 "메시지"를 한 번에 암호화할 
% 필요가 없는 구성 3.17에 대응하는 것으로 볼 수 있습니다. 위는 서로 다른 메시지의 암호화/복호화 사이의 
% 상태를 유지하기 위해 송신자와 수신자가 필요한 상태 암호화의 예입니다. 상태 암호화에 적합한 CPA 보안 
% 개념을 정의하고 기본 스트림 암호가 안전하다면 위의 체계가 해당 정의를 충족하는지 증명할 수 있습니다.
% 동기화 모드의 경우 스트림 암호는 IV를 사용할 필요가 없습니다. 또한 송신자에서 수신자로의 전체 통신은 
% 암호화되는 메시지의 전체 길이와 정확히 일치하므로 암호문 확장은 없습니다.

(un-synchronized mode) When a stream cipher does take an IV, it can be used to
construct a stateless encryption scheme:
\begin{itemize}
  \item $\gen$: on input $1^n$, choose a uniform $\key \in \binset{n}$ and
        output it.
  \item $\enc$: on input a key $\key$ and a message $\pt \in \binset{*}$,
        choose uniform $\iv \in \binset{n}$, and output the ciphertext
        $\cyc{\iv, \getbits_1(\stinit(\key, iv), 1^{|pt|}) \xor \pt}$.

  \item $\dec$: on input a key $k$ and a ciphertext $\cyc{iv, \ct}$, output
        the message
        $\pt := \getbits_1(\stinit(\key, \iv), 1^{|\ct|}) \xor \ct$.
\end{itemize}
% 여기서 가장 큰 장점은 암호 체계가 임의 길이의 메시지를 직접 처리한다는 것임을 강조합니다.

\newpage

\subsection*{Block Ciphers and Block-Cipher Modes of Operation}
블록 암호 $\bc_k: \binset{\blen} \to \binset{\blen}$ 은 (strong) PRP 의 다른 이름이다.
블록 암호는 임의의 길이를 지원하는 PRP와 다르게 특정한 키/블록 길이 $\klen, \blen$을 사용한다는
점이다. 이 장에서는 블록 암호의 운영 모드 5개를 소개한다. 단순화를 위해 모든 메시지 벡터를
$\pt_i \in \binset{\blen}$에 대해 $\vec{\pt} := (\pt_1, \pt_2, \cdots, \pt_l)$로
정의한다. (실제 메시지의 길이가 $\blen$의 배수가 아니더라도 패딩으로 해결 가능하므로, 이 가정은
일반성을 잃지 않는다.)

\begin{itemize}
  \item ECB: $\ct_i := \bc_k(\pt_i)$ and $\pt_i := \ibc_k(\ct_i)$.
  \item CBC: $\ct_i := \bc_k(\ct_{i-1} \xor \pt_i)$ and
        $\pt_i := \ibc_k(\ct_i) \xor \ct_{i-1}$.
        $\ct_0 := \iv \in \binset{n}$ is uniform. $i = 1,2, \cdots, l$.
  \item OFB:
  \item CFB:
  \item CTR: $y_i = \bc_k(\iv || \cyc{i})$. $\iv \in \binset{3n/4}$ is uniform.
        $i = 1, 2, \cdots, l \le 2^{n/4}$.
        \memo{일반적인 $\iv$의 크기는?}
\end{itemize}

\newline

\begin{theorem} \label{thm:ecb_is_not_ind_pass}
  ECB Mode is not \textsf{IND-PASS} secure.
\end{theorem}

\begin{proof}
  \memo{안 어려우므로 생략.}
  To prove something is not secure we need to exhibit an attack within the
  model. The attack on ECB mode is very simple:

  Let $\vec{0}$ denote the block of all zeros, and $\vec{1}$ denote the block
  of all ones. Call the encryption oracle with $\vec{m_0}$ = $\vec{0}||\vec{1}$
  and $\vec{m_1}$ = $\vec{1}||\vec{1}$ The challenge ciphertext $\vec{c}^*$ is
  returned which is the encryption of $\vec{m}_b$, for the hidden bit $b$. The
  challenge ciphertext consists of two blocks $\vec{c}_0$ and $\vec{c}_1$. If
  $\vec{c}_0 \neq \vec{c}_1$ then output $b' = 0$, else return $b' = 1$.
\end{proof}

\newline

\begin{lemma} \label{lem:collision}
  Fix a positive integer $N$, and say $q \le \sqrt{2N}$ elements
  $y_1, \cdots, y_q$ are chosen uniformly and independently from a set of size
  $N$. Then
  \begin{equation}
    \frac{q \cdot (q-1)}{4N} \le \coll(q,N) \le \frac{q \cdot (q-1)}{2N}
  \end{equation}
\end{lemma}

\begin{proof}
  \memo{일단 생략.}
\end{proof}

\newline

\begin{theorem} \label{thm:cbc_is_ind_cpa}
  If $\prp$ is a pseudorandom permutation, then CBC mode
  is \textsf{IND-CPA} secure.
\end{theorem}

\begin{proof}
  $\prp$는 PRP, $\rp$는 RP라고 하고, $q(n)$을 임의의 PPT 구분자 $\adv$가 암호화
  오라클에 질의하는 최대 다항 횟수라고 하자. 마지막으로 $\cbc[\prp]$를 $\prp$를 블록암호로
  사용하는 CBC 운영 모드, $\cbc[\rp]$를 $\rp$를 블록암호로 사용하는 CBC 운영 모드라고 하자.
  증명은 다음과 같은 단계로 진행한다.
  \begin{enumerate}
    \item 먼저, 다음을 만족하는 negligible 함수 $\negl$가 있음을 보인다.
          \begin{equation} \label{equ:pf_cbc_1}
            \abs {
              \pr{\expcpa{\adv, \cbc[\prp]}(n) = 1} -
              \pr{\expcpa{\adv, \cbc[\rp]}(n) = 1}
            } \le \negl(n).
          \end{equation}
    \item 이후 다음을 만족함을 보인다.
          \begin{equation} \label{equ:pf_cbc_2}
            \pr{\expcpa{\adv, \cbc[\rp]}(n) = 1}
            \le \frac{1}{2} + \frac{q^2}{2^{n + 1}}.
          \end{equation}
    \item 위의 두 식을 통해 다음을 만족함을 알 수 있고, $q$는 다항식이므로, 증명이 완성된다.
          \begin{equation}
            \pr{\expcpa{\adv, \cbc[\prp]}(n) = 1}
            \le \frac{1}{2} + \frac{q^2}{2^{n + 1}} + \negl(n).
          \end{equation}
  \end{enumerate}

  첫 번째 식 \ref{equ:pf_cbc_1}을 증명하기 위해, 실험 \ref{fig:exp_ind_cpa_cbc}을
  구성하자. 만약 $\orc = \prp$이라면, $\adv$ 관점에서 볼 때, 이 실험은
  $\expcpa{\adv, \cbc[\prp]}$ 실험과 같다. 따라서 다음이 성립한다.
  \begin{equation} \label{equ:pf_cbc_prp}
    \pr{\advp^{\prp}(1^n) = 1} = \pr{\expcpa{\adv, \cbc[\prp]}(n) = 1}.
  \end{equation}
  비슷한 방식으로, 만약 $\orc = \rp$라면, 이 실험은 $\expcpa{\adv, \cbc[\rp]}$ 실험과 같다.
  따라서 다음이 성립한다.
  \begin{equation} \label{equ:pf_cbc_rp}
    \pr{\advp^{\rp}(1^n) = 1} = \pr{\expcpa{\adv, \cbc[\rp]}(n) = 1}.
  \end{equation}
  $\prp$는 PRP이므로, 다음을 만족하는 negligible 함수 $\negl$가 존재한다.
  \begin{equation}
    \abs{
      \pr{\advp^{\prp}(1^n) = 1} - \pr{\advp^{\rp}(1^n) = 1}
    } \le \negl(n).
  \end{equation}
  이 식과 위에서 보인 두 식 \ref{equ:pf_cbc_prp}과 \ref{equ:pf_cbc_rp}을 통해 증명하기로한
  첫 번째 식을 증명할 수 있다.

  \begin{figure}[ht]
    \centering
    \begin{tikzpicture}
      [inner sep=0.5cm, every text node part/.style={align=center}]
      % experiment players and what players are doing
      \node at (0, 0) (chl) {challenger $\chl$};
      \node at (6, 0) (adv_p) {adversary $\advp$};
      \node at (12, 0) (adv) {adversary $\adv$};

      % draw lines
      \draw[<->] (chl) -- (adv_p)
      node[above, midway, inner sep=0.1cm] {\textsf{PRP}};
      \draw[<->] (adv_p) -- (adv)
      node[above, midway, inner sep=0.1cm] {\textsf{IND-CPA}};

      % challenger C
      \node at (0, -1) (key_gen) {$k \samp \gen(1^n)$};
      \node at (0, -2.5) (samp_orc) {
        $b \unisamp \binset{}$ \\
        $\orc := \prp$, if $b = 1$ \\
        $\orc := \rp$, otherwise
      };
      \node at (0, -4) (orc) {$c_i \samp \orc(x_i)$};
      \node at (0, -8) (orc2) {$c_i \samp \orc(x_i)$};
      \node at (0, -11) (orc3) {$c_i \samp \orc(x_i)$};
      \node at (0, -13) (get_bp2) {$b'$};
      \node at (0, -14) (return) {return $b' \issame b$};

      % adversary B
      \node at (6, -4) (cbc) {
        $\iv \unisamp \binset{n}$ \\
        $x_i := m_i \xor \ct_{i - 1}$ \\
        $(c_0 := \iv, i = 1, \cdots, l)$ \\
        $\vec{\ct} = (\iv, \ct_1, \cdots, \ct_{\blen})$
      };
      \node at (6, -6) (samp_b) {$b^* \unisamp \binset{}$};
      \node at (6, -8) (cbc2) {
        $\iv \unisamp \binset{n}$ \\
        $x_i := m_{b^*, i} \xor \ct_{i - 1}$ \\
        $(c_0 := \iv, i = 1, \cdots, l)$ \\
        $\vec{\ct} = (\iv, \ct_1, \cdots, \ct_{\blen})$
      };
      \node at (6, -11) (cbc3) {
        $\iv \unisamp \binset{n}$ \\
        $x_i := m_i \xor \ct_{i - 1}$ \\
        $(c_0 := \iv, i = 1, \cdots, l)$ \\
        $\vec{\ct} = (\iv, \ct_1, \cdots, \ct_{\blen})$
      };
      \node at (6, -13) (get_bp) {$b'$};

      % adversary A
      \node at (12, -4) (chs_m) {choose $\vec{m}$};
      \node at (12, -8) (chs_m0m1) {choose $\vec{m}_0, \vec{m}_1$};
      \node at (12, -11) (chs_m_2) {choose $\vec{m}$};
      \node at (12, -13) (chs_bp) {choose $b' \in \binset{}$};

      % draw lines: B <-> A
      \draw[blue, <->] (cbc) -- (chs_m)
      node [midway, above, inner sep=0.1cm] {query $\vec{m}$}
      node [midway, below, inner sep=0.1cm] {response $\vec{c}$};
      \draw[<->] (cbc2) -- (chs_m0m1)
      node [midway, above, inner sep=0.1cm] {send $\vec{m_0}, \vec{m_1}$}
      node [midway, below, inner sep=0.1cm] {challenge $\vec{c}$};
      \draw[blue, <->] (cbc3) -- (chs_m_2)
      node [midway, above, inner sep=0.1cm] {query $\vec{m}$}
      node [midway, below, inner sep=0.1cm] {response $\vec{c}$};
      \draw[<-] (get_bp) -- (chs_bp);

      % draw lines: C <-> B
      \draw[red, <->] (orc) -- (cbc)
      node [midway, above, inner sep=0.1cm] {query $x_i$}
      node [midway, below, inner sep=0.1cm] {response $c_i$};
      \draw[red, <->] (orc2) -- (cbc2)
      node [midway, above, inner sep=0.1cm] {query $x_i$}
      node [midway, below, inner sep=0.1cm] {response $c_i$};
      \draw[red, <->] (orc3) -- (cbc3)
      node [midway, above, inner sep=0.1cm] {query $x_i$}
      node [midway, below, inner sep=0.1cm] {response $c_i$};
      \draw[<-] (get_bp2) -- (get_bp);
    \end{tikzpicture}
    \caption{\textsf{IND-CPA} experiment of private-key encryption scheme with
      CBC mode}
    \label{fig:exp_ind_cpa_cbc}
  \end{figure}

  다음으로 두 번째 식 \ref{equ:pf_cbc_2}을 증명한다. $\repeat$을 $\adv$가 암호화 오라클에
  질의할 때, 초기화 벡터 $\iv$가 적어도 한 번 이전과 같은 값을 사용하는 사건이라고 하자. $\adv$가
  모든 질의을 마쳤을 때, 다음의 두 사건이 있을 수 있다.
  \begin{itemize}
    \item ($\repeat$이 일어나지 않은 사건, 즉, $\neg\repeat$) 만약 $\orc = \rp$라면,
          $\iv$는 균등하게 선택되었으므로, 블록암호의 출력도 균등분포를 따른다. 따라서 $\adv$의
          구분 성공 확률은 정확히 $1/2$이다. 정리하면 다음과 같다.
          \begin{equation}
            \pr{\expcpa{\adv, \cbc[\rp]}(n) = 1 \when \neg\repeat}
            \cdot \pr{\neg\repeat} = \frac{1}{2}.
          \end{equation}

    \item ($\repeat$이 일어난 사건) $\iv$를 균등하게 독립적으로 선택하므로, 보조정리
          \ref{lem:collision}에 의해, 이 사건이 발생할 확률은 다음과 같다.
          \begin{equation}
            \pr{\repeat} \le \frac{q^2}{2^{n + 1}}.
          \end{equation}
  \end{itemize}
  위의 두 식을 이용하여, 다음과 같이 두 번째 식을 증명할 수 있고, 이로써 증명은 완성된다.
  \begin{equation}
    \begin{split}
      &\pr{\expcpa{\adv, \cbc[\rp]}(n) = 1} \\
      &= \underbrace{
        \pr{\expcpa{\adv, \cbc[\rp]}(n) = 1 \when \neg\repeat}
        \cdot \pr{\neg\repeat}
      }_{= \frac{1}{2}}
      + \underbrace{
        \pr{\expcpa{\adv, \cbc[\rp]}(n) = 1 \when \repeat}
        \cdot \pr{\repeat}
      }_{\le \pr{\repeat} \le \frac{q^2}{2^{n + 1}}} \\
      &\le \frac{1}{2} + \frac{q^2}{2^{n + 1}}.
    \end{split}
  \end{equation}
\end{proof}

\newline

\begin{theorem} \label{thm:ctr_is_ind_cpa}
  If $\prf$ is a pseudorandom function, then CTR mode is \textsf{IND-CPA}
  secure for multiple encryptions.
\end{theorem}

\begin{proof}
  정리 \ref{thm:cbc_is_ind_cpa}과 유사하므로, 생략.
\end{proof}

\subsection*{Nonce-Based Encryption}

% 지금까지 우리는 개인 키 암호화에 대한 한 가지 특정 구문을 고려했습니다. 여기서는 일부 맥락에서 유용한 
% 개인 키 암호화를 공식화하는 대안적인 방법을 살펴봅니다. 구체적으로, 우리는 암호화 및 복호화 알고리즘이 
% 추가로 논스를 입력으로 받아들이는 논스 기반 암호화의 개념을 고려합니다. ("논스"는 한 번만 사용되고 결코 
% 반복되지 않아야 하는 값을 나타냅니다.) 논스 기반 암호화의 구문은 논스가 어디에서 왔는지를 지정하지 
% 않습니다. 실제로, 논스는 동일한 논스가 한 번 이상 암호화하는 데 절대 사용되지 않도록 보장해야 하는 일부 
% 상위 수준 응용 프로그램에 의해 제공됩니다. 예를 들어, 논스는 카운터일 수도 있고, 현재 시간일 수도 
% 있습니다.

\newline

\begin{definition} \label{def:nonce}
  A nonce-based (private-key) encryption scheme consists of PPT algorithms
  $(\gen, \enc, \dec)$ such that:
  \begin{itemize}
    \item $\gen$ takes as input $1^n$ and outputs a key $k$ with $|k| \ge n$.
    \item $\enc$ takes as input a key $k$, a nonce $\nonce \in \binset{*}$, and
          a message $m \in \binset{*}$, and outputs a ciphertext $c$.
    \item $\dec$ takes as input a key $k$, a nonce $\nonce$, and
          a ciphertext $c$, and outputs a message $m$ or $\err$.
  \end{itemize}
  We require that for every $n$, every $k$ output by $\gen(1^n)$, every
  $\nonce \in \binset{*}$, and every $m \in \binset{*}$, it holds that
  $\dec(\nonce, \enc(\nonce, m)) = m$.
\end{definition}

% 일부 논스 기반 암호화 체계는 특정 길이의 논스만 지원합니다. 이 경우에 대해 논의하는 모든 정의를 쉽게 
% 적용할 수 있습니다.

% 논스 기반 암호화에 대한 보안은 이전에 본 정의를 적절히 적용하여 정의할 수 있으며, 구체적으로 설명하기 
% 위해 다중 암호화에 대한 CPA-secure 개념을 적용합니다. 여기서 고려하는 실험은 이전 정의에서 고려한 
% 실험과 개념적으로 동일하며, 특히 공격자에게 두 개의 메시지를 받아들이고 "왼쪽" 또는 "오른쪽" 메시지를 
% 암호화하는 LR 오라클에 대한 액세스를 다시 제공합니다. 여기서 차이점은 공격자가 절대 논스를 반복하지 
% 않는다는 제약 조건에 따라 암호화 중에 사용되는 논스를 지정할 수도 있다는 것입니다.

\newline

\begin{figure}[h] \label{fig:exp_ind_cpa_nonce}
  \centering
  \begin{tikzpicture}[inner sep=0.5cm]
    % experiment players and what players are doing
    \node () {};
    \node at (0, 0) (chl) {challenger $\chl$};
    \node at (9, 0) (adv) {adversary $\adv$};

    % draw lines
    \draw[<->] (chl) -- (adv)
    node[above, midway, inner sep=0.1cm] {\textsf{IND-CPA} experiment};

    % challenger C
    \node at (0, -1) (key_gen) {$k \samp \gen(1^n)$};
    \node at (0, -2) (samp_b) {$b \unisamp \binset{}$};
    \node at (0, -3) (orc) {$c = \enc(m, \nonce)$};
    \node at (0, -4) (get_m0m1) {$m_0, m_1$};
    \node at (0, -5) (get_c) {$c = \enc(m_b, \nonce)$};
    \node at (0, -6) (orc2) {$c = \enc(m, \nonce)$};
    \node at (0, -7) (get_bp) {$b'$};
    \node at (0, -8) (return) {return $b \issame b'$};

    % adversary A
    \node at (9, -3) (chs_m) {choose $m, \nonce$};
    \node at (9, -4) (chs_m0m1) {choose $m_0, m_1, \nonce$};
    \node at (9, -5) (get_c_2) {$c$};
    \node at (9, -6) (chs_m_2) {choose $m, \nonce$};
    \node at (9, -7) (chs_bp) {choose $b' \in \binset{}$};

    % draw lines
    \draw[blue, <->] (orc) -- (chs_m)
    node [midway, above, inner sep=0.1cm] {query $(m, \nonce)$}
    node [midway, below, inner sep=0.1cm] {response $c$};

    \draw[<-] (get_m0m1) -- (chs_m0m1);
    \draw[->] (get_c) -- (get_c_2);

    \draw[blue, <->] (orc2) -- (chs_m_2)
    node [midway, above, inner sep=0.1cm] {query $(m, \nonce)$}
    node [midway, below, inner sep=0.1cm] {response $c$};

    \draw[<-] (get_bp) -- (chs_bp);
  \end{tikzpicture}
  \caption{\textsf{IND-CPA} experiment of nonce-based private-key encryption
    scheme}
\end{figure}

% IND-CPA secure 논스 기반 암호화 방식을 얻기 위해 CTR 모드를 수정하는 것은 쉽습니다. 암호화할 때 
% IV는 이제 균일하게 선택되는 것이 아니라 입력으로 제공되는 논스와 동일하게 설정됩니다. CPA-보안은 반복이 
% 발생할 수 없다는 사실을 사용하여 정리 3.33의 증명과 정확히 일치할 수 있습니다. (상대가 논스를 반복하는 
% 것이 허용되지 않기 때문에). 실제로 여기서 얻은 구체적인 보안 경계는 여기서 반복이 발생할 수 없기 때문에 
% 정리 3.33의 증명에서 얻은 것보다 정확하게 더 좋습니다. 물론 이는 암호화 방식을 사용하는 응용 프로그램이 
% 논스가 절대 반복되지 않도록 보장한다는 가정에 기초합니다. 우리는 논스 기반 암호화 방식이 결정론적임에도 
% 불구하고 CPA-보안이 될 수 있음을 알 수 있습니다. 여기서는 암호화를 위한 대체 구문을 고려하고 있기 
% 때문에 이것은 정리 3.20과 모순되지 않습니다.

% 특히 임의의 논스 기반 암호화 체계는 단순히 무작위로 논스를 선택함으로써 "표준" 암호화 체계로 변환될 수 
% 있기 때문에 논스 기반 암호화를 사용하면 무엇을 얻을 수 있는지 궁금할 수 있습니다. 이 질문에 대한 몇 
% 가지 답이 있습니다. 우선 CPA 보안 논스 기반 암호화는 고품질 무작위성을 생성하는 데 비용이 많이 들거나 
% 불가능한 설정에서 유용합니다. 이러한 경우 균일하게 논스를 생성하는 것보다 카운터를 논스로 사용하는 것이 
% 훨씬 쉬울 수 있습니다. 다소 유사하게, 예를 들어 매우 적은 수의 메시지만 암호화되는 짧은 논스를 사용하는 
% 것이 적절한 설정이 있을 수 있습니다. 이러한 시나리오에서 균일하게 논스를 선택하면 허용할 수 없을 정도로 
% 높은 확률로 반복되는 논스가 발생할 수 있습니다. 마지막으로, 우리는 이미 균일한 논스를 선택하는 것보다 
% 비반복적인 논스를 시행함으로써 더 엄격한 구체적인 보안 경계를 얻을 수 있음을 관찰했습니다.

\begin{theorem} \label{thm:nonce_cbc_is_nt_ind_cpa}
  With a nonce as the IV, CBC mode is not IND-CPA secure.
\end{theorem}

\begin{proof}
  \memo{안 어려우므로 생략.}
  Let $\vec{0}$ be the all-zero block and $\vec{1}$ be the all-one block. The
  attack on the \textsf{IND-CPA} security is as follows:
  Send the message $\vec{0}$ with the nonce $\vec{0}$ to the
  encryption oracle. The adversary obtains the ciphertext $\vec{0}||c$ in
  return, where $c = \enc(\vec{0}, \vec{0})$. Now send the messages
  $m_0 = \vec{0}$ and $m_1 = \vec{1}$ to the encryption oracle, with nonce
  $\vec{1}$. Notice this is a new nonce and so the encryption is allowed in the
  game. Let $\vec{1}||c^*$ be the returned ciphertext. If $c^* = c$ then return
  $b' = 1$, else return $b' = 0$.
\end{proof}

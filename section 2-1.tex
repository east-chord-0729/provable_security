\section{Definition}

\begin{definition}
    A public-key encryption scheme is a triple of probabilistic polynomial-time algorithms $\schf$ such that:
    \begin{itemize}
        \item $(pk, sk) \samp \gen$
        \item $c \samp \encp{m}$
        \item $m \samp \decp{c}$
    \end{itemize}
\end{definition}

\subsection{Security against Chosen-Plaintext Attacks}

\imgdraw{\exppkimgs}{indistinguishability experiment of public-key encryption scheme}{exppk}

\begin{definition}
    A public-key encryption scheme $\sch$ has indistinguishable encryptions in the presence of an eavesdropper if for all ppt distinguishers $\adv$ there is a negligible function $\negl$ such that

    $$\abs{\pr{\pubeav{\adv}(n, 1) = 1} - \pr{\pubeav{\adv}(n, 0) = 1}} \le \negl$$
\end{definition}

위의 정의와 정의 3.8의 주요 차이점은 여기서 $\adv$는 공개 키 $pk$를 부여받았다는 것입니다. 또한 \emph{$\adv$는 이 공개 키를 기반으로 자신의 메시지 $m_0$과 $m_1$을 선택}할 수 있습니다. 앞서 설명한 대로 상대방이 수신자의 공개 키를 알고 있다고 가정하는 것이 합리적이기 때문에 공개 키 암호화의 보안을 정의할 때 이는 필수적입니다.

구분자 $\adv$에게 공개 키 $pk$를 제공하는 것은 \emph{$\adv$에게 무료로 암호화 오라클에 액세스할 수 있게 해줍니다}. 결론적으로, 공개 키 암호의 보안 정의는 \emph{CPA-보안과 동일}하며, 유일한 차이점은 공격자에게 해당 실험에서 공개 키가 제공된다는 것입니다.

\begin{proposition}
    If a public-key encryption scheme has indistinguishable encryptions in the presence of an eavesdropper, it is CPA secure.
\end{proposition}

\begin{theorem}
    No deterministic public-key encryption scheme is CPA-secure.
\end{theorem}